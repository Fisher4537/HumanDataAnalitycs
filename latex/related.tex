% !TEX root = CartaPerali_Report.tex

\section{Related Work}
\label{sec:related_work}

\noindent 
Deep Neural Networks are the most powerful tool for the KWS task as they allow to reach state-of-art performances. {\it [Jansson_Patrick.pdf]} While traditional machine learning techniques e.g. Support Vectors Machines (SVM), require hand-made features to reach optimal results, Neural Networks are capable of succesfully train from more raw data. {\it [1710.10361]} Their deployment of course is not new. Convolutional Neural Networks (CNN) are the standard tool for small-footprint KWS since they have a relatively standard architechture which is easy to implement and tune. Additionally, they are available in several of the most used Deep Learning frameworks. {\it [1808.08929] } introduces a novel attention-based Recurrent Neural Networks (RNN) architecture designed to recognize simple speech commands, while still generating a lightweight model that can be loaded in mobile devices and run locally. The proposed architecture uses raw {\it .wav} files as inputs, computes mel-scale spectrogram using a non-trainable Keras layer, extracts short and long-term dependencies and uses an attention mechanism to pinpoint which region has the most useful information, that is then fed to a sequence of dense layers. In {\it [186]} the Key-word/Garbage model is used. A Hidden Markov Model (HMM) is constructed for each keyword. In addition, garbage HMMs are constructed to represent all non-keywords. A Viterbi decoding is then used to find the best path that explains the input speech. Obtained results show very low false reject rates but, very high false alarm rates, since theg arbage model can represents all non-keywords properly.

%
