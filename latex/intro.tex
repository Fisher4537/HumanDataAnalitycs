% !TEX root = template.tex

\section{Introduction}
\label{sec:introduction}

\begin{remark}
\textbf{Paper contribution.} In this paper we describe how we implemented a keyword spotting system investigating different architechtures and learning techniques. 
\end{remark}

\begin{remark}
\textbf{Dataset description.} We trained our systems through the "{\it speech\_commands}" dataset. It contains 105,829 utterances of 35 distinct words coming from 2,618 different persons, each one having the duration of at most one second, stored in {\it .wav} format files. The contained audio allow to reflect the trigger phrase task that a keyword spotting system aims at achieving and cope with noisy environments, poor quality recording tools or people taking in a natural way. 
\end{remark}

\red{[Maximum length for the whole report is 9 pages. Abstract, introduction and related work should take max two pages.]}\\

\noindent \textbf{Recommended structure for the intro:} you may use the following structure. 
\begin{itemize}
\item \textbf{General (short) intro:} Several voice interfaces (i.e. Alexa, Siri, Cortana) rely on keyword spotting, for the detection of wake-up words that determine the beginning of an interaction. These systems are continuously listening for an audio input from the embedded microphones and, once a likely trigger phrase is detected, the interaction begins. 
\item \textbf{Put the problem into perspective:}  Our work addressed keyword spotting as a closed-set classification problem, as most of the existing solutions. We start by deploying a CNN-based architecture playing on different data preprocessing approaches and hyperparameters tuning. {\it to be continued}
\item \textbf{Present the paper contribution:} A third paragraph were you state what you do in the paper, this should also be concisely written. A good rule of thumb is to make it max 10/15 lines. Here, you should state up front:
\begin{enumerate}
\item \textbf{problem}: the problem at stake, 
\item \textbf{relevance}: the relevance and timeliness of what you propose, 
\item \textbf{approach}: the technique/approach you use, possibly underlying its novelty, efficiency, 
\item \textbf{value}: underline the value/novelty of your proposal referencing (recent) papers from the literature,
\item \textbf{applicability}: tell the reader how she/he can take advantage of your work, e.g., how your work/results can be reused/exploited to achieve further scientific, technical or practical (integrated into products?) goals.
\end{enumerate}
\item \textbf{Summary of contributions:} After this, you may want to provide an itemized list to summarize the paper contributions. Rule of thumb: from three to six items, from three to four lines each.
\item \textbf{Closing (paper structure):}In Section II we describe the state of the art, the system and data models are respectively presented in Sections~III and~IV. The proposed signal processing technique is detailed in Section~V and its performances evaluation is carried out in Section~VI. Concluding remarks are provided in Section~VII.''
\end{itemize}


